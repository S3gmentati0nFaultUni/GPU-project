\documentclass[a4paper,10pt]{article}
\usepackage{src/preamble}
\usepackage[
	backend=biber,
	maxalphanames=10,
]{biblatex}
\bibliography{bibliography.bib}

\begin{document}

\noindent
\begin{center}
	\textbf{{PARALLEL SOLUTION FOR THE MST PROBLEM}} \\
\end{center}

\noindent
\textbf{Author: Alessandro Biagiotti} \hfill \textit{Milan university}
\\

%\noindent 
%\textbf{Research Supervisor: D. Duck,} \textit{University of Bristol, Bristol, U.K.}
%\\

\noindent
\textbf{ABSTRACT:}
\\

\noindent
\textbf{KEYWORDS:}
\\

\noindent
\textbf{STATEMENT OF ORIGINALITY:}

\noindent
\phantomsection
\makeatletter\def\@currentlabel{\texttt{(I)}}\makeatother
\label{sec:intro}
\textbf{INTRODUCTION TO THE PROBLEM:}
\\
Lorem ipsum

\noindent
\phantomsection
\makeatletter\def\@currentlabel{\texttt{(I)}}\makeatother
\label{sec:intro}
\textbf{IMPLEMENTAITON ROADMAP:}
\\
\begin{enumerate}
	\item Started with a naive resolution of the problem using Prim's algorithm with no additional enrichment for the CPU code
	\item Written a naive solver for the problem that leverages the GPU capacities
	\item The solution originally written for the CPU could not keep up with the GPU solver I therefore implemented a small Heap class that was able to move the CPU solver to the next level
	\item The GPU implementation found itself struggling essentially because of the nature of the problem and the not-so-efficient implementation
\end{enumerate}

\printbibliography

\clearpage

\end{document}
