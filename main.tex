\documentclass[a4paper,10pt]{article}
\usepackage[utf8]{inputenc}
\usepackage{src/preamble}
\usepackage[
	backend=biber,
	maxalphanames=10,
]{biblatex}
\bibliography{bibliography.bib}

\begin{document}

\noindent
\begin{center}
	\textbf{{PARALLEL SOLUTION FOR THE MST PROBLEM}} \\
\end{center}

\noindent
\textbf{Author: Alessandro Biagiotti} \hfill \textit{Milan university}
\\

\noindent
\textbf{ABSTRACT:}
\\

\noindent
\textbf{KEYWORDS:}
\\

\noindent
\textbf{STATEMENT OF ORIGINALITY:}

\bigskip

\phantomsection
\makeatletter\def\@currentlabel{\texttt{(I)}}\makeatother
\label{sec:intro}
\noindent
\textbf{INTRODUCTION TO THE PROBLEM:}
\\
The Minimum Spanning Tree Problem (\mstp for short) is a problem that has been studied for years and to this day finds many real world applications,to name a few:
\begin{enumerate}
	\item It's possible to find different \mst-based techniques that can be used to do image segmentation~\cite{maze-generation}~\cite{mst-segmentation-heuristic}.
	\item Finding the \mst allows us to solve the Clustering problem, all we need to do is to compute an \mst and then drop the $k - 1$ most expensive edges of the \mst~\cite{mst-applications}.
	\item Finding the \mst is an important step of Christofides's algorithm, which is a 2-approximation for the widely known Travelling Salesman Problem (\textsc{Tsp} for short)~\cite{tsp-christofides}.
\end{enumerate}

Many different solutions have been found for the problem in the years, the more basic implementations were Prim's~\cite{prim-algorithm} and Kruskal's~\cite{kruskal-algorithm}.

Prim's algorithm is the simplest solver for \mstp, given a graph $G= (V, E)$, let us suppose that the size of the graph $|V|$ is $n$ and the number of edges $|E|$ is $m$, then, based on the implementation, the computational complexity ranges between the following values:
\begin{itemize}
	% TODO Fix the ï not showing properly
	\item Naïve implementation: $\O(n^2)$
	\item Binary heap and adjacency list: $\O(m\log{n})$
	\item Fibonacci heap and adjacency list: $\O(m + n\log{n})$
\end{itemize}
In my implementation I have implemented both the Naïve version and the one using a binary heap, more on this in the next section.

Kruskal's algorithm is the alternative to Prim's algorithm and it works in time $\O(m\log{m})$ if implemented using classical data structures.

\brka's algorithm is a solver for the \mstp based on forests, a more verbose version of the algorithm shown in~\cite{boruvka-pseudocode} can be seen in~\ref{algo:boruvka-pseudocode}
\begin{algorithm}
	\caption{\brka's algorithm}
	\label{algo:boruvka-pseudocode}
	\begin{algorithmic}[1]
		\REQUIRE An undirected, connected and weighted graph $G$, a set of empty edges $T$
		\ENSURE An \mst$T$ built on graph $G$
		\WHILE{vertices in $G$ connected by $T$ are disjoint}
			\STATE start with an empty set of edges $E$
			\FOR{every connected component}
				\STATE start with an empty set of edges $S$
				\FOR{every vertex $v$ in the component}
					\STATE add the cheapest edge going from $v$ to any other component to $S$
				\ENDFOR
				\STATE add the cheapest edge in $S$ to $E$
			\ENDFOR
			\STATE add the set of edges from $E$ to $T$ 
		\ENDWHILE
		\STATE\RETURN $T$
	\end{algorithmic}
\end{algorithm}

Let us suppose that the size of the graph $|V|$ is $n$ and the number of edges $E$ is $m$, then the computational complexity of the pseudocode shown in~\ref{algo:boruvka-pseudocode} is $\O(m\log{n})$. This algorithm has been later reused in combination with other techniques to compute a solution for the \mstp very efficiently~\cite{boruvka-ackermann}~\cite{karger-klein-tarjan}. As for the solutions previously proposed for the sequential resolution of the problem there is different parallel implementations that are based on \brka's solution, this is due to the inherent parallelizability of the algorithm.

Regardless of the implementation the core of the algorithm can be identified in the following four steps as identified in~\cite{boruvka-steps}:
\begin{enumerate}
	\item\label{item:first-step} (\textit{choose lightest}) for every vertex the lightest edge is chosen in parallel.
	\item\label{item:second-step} (\textit{find root}) for every vertex we find the root of the tree to which it belongs\footnote{It's important to keep in mind that vertices are actually supervertices.}.
	\item\label{item:third-step} (\textit{rename vertices}) Since the graph after~\ref{item:fourth-step} is compressed, which means that the number of vertices decreases, then the roots in the graph need to be renamed. 
	\item\label{item:fourth-step} (\textit{graph compression}) The graph undergoes a compression step, the result a compressed version of the graph containing only the roots identified in~\ref{item:second-step} and the edges that connect a component $C_i$ to any other component $C_j$, therefore the graph only loses the edges contained in the original graph.
\end{enumerate}

The specific implementation that I chose to follow is shown in~\cite{generic-he-boruvka} and is summarized by algorithm~\ref{algo:boruvka-parallel}

\begin{algorithm}
	\caption{\brka's algorithm}
	\label{algo:boruvka-parallel}
	\begin{algorithmic}[1]
		\REQUIRE An undirected, connected and weighted graph $G(V, E)$
		\ENSURE An \mst$T$ built on graph $G$
		\WHILE{$|V| > 1$}
			\STATE find the minimum edge per vertex (\ref{item:first-step})
			\STATE remove mirrored edges
			\STATE initialize the colors
			\WHILE{the coloring hasn't converged}
				\STATE color propagation (\ref{item:second-step})
			\ENDWHILE
			\STATE create new vertex ids (\ref{item:third-step})
			\STATE do graph compression (\ref{item:fourth-step})
		\ENDWHILE
		\STATE\RETURN $G$
	\end{algorithmic}
\end{algorithm}

The graph returned in the last step is a single supervertex containing the \mst.

In the following sections I will be going through the overall implementation roadmap~\ref{sec:implementation-roadmap}, the implementation of the scan procedure~\ref{sec:parallel-scan}, an analysis of the algorithm's performance~\ref{sec:performance-analysis} and in the last section I will be going over an analysis of some of the problems identified during the implementation and some further development that could be made to make the implementation better~\ref{sec:final-thoughts}. 

\bigskip
\phantomsection
\makeatletter\def\@currentlabel{\texttt{(I)}}\makeatother
\label{sec:implementation-roadmap}
\noindent
\textbf{IMPLEMENTAITON ROADMAP:}
\\
\begin{enumerate}
	\item Started with a naive resolution of the problem using Prim's algorithm with no additional enrichment for the CPU code
	\item Written a naive solver for the problem that leverages the GPU capacities
	\item The solution originally written for the CPU could not keep up with the GPU solver I therefore implemented a small Heap class that was able to move the CPU solver to the next level
	\item The GPU implementation found itself struggling essentially because of the nature of the problem and the not-so-efficient implementation
	\item Changed the GPU implementation to use an hybrid of CPU code and GPU kernels, the CPU code is used in settings such as the scan, which is more efficient than using a naive GPU kernel, as well as the contraction operation
	\item Reimplemented the contraction operation using a GPU kernel and I obtained a minor speedup. The GPU performance is still far inferior if compared with the efficient CPU version that I talked about earlier
	\item Reimplemented the scan operation to work in a work efficient way, as a source I used the preprefix file and implemented the various sequential operations in a naïve way.
	\item Reimplemented the fifth step of the algorithm, changed the logic from topological (one thread per vertex) to data centered (one thread per edge) each thread goes through a binary search of the origin of the edge in the cumDegs array.
\end{enumerate}

\bigskip
\phantomsection
\makeatletter\def\@currentlabel{\texttt{(II)}}\makeatother
\label{sec:parallel-scan}
\noindent
\textbf{EXCLUSIVE PARALLEL SCAN IMPLEMENTATION:}
\\

\bigskip
\phantomsection
\makeatletter\def\@currentlabel{\texttt{(III)}}\makeatother
\label{sec:performance-analysis}
\noindent
\textbf{PERFORMANCE ANALYSIS:}
\\

\bigskip
\phantomsection
\makeatletter\def\@currentlabel{\texttt{(IV)}}\makeatother
\label{sec:final-thoughts}
\noindent
\textbf{FINAL THOUGHTS:}
\\

\clearpage

\printbibliography

\end{document}
